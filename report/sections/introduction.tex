\section{Introduction}
The resilience and efficiency of transportation networks are critical to societal infrastructure, especially in interconnected systems like the European railway network, which facilitates daily commuting, economic trade, and tourism.

In this work, first I will construct a model by using the passenger train data available on the online platform \emph{Trainline}. The resulting network can be thought of as the network of all the train routes that European train companies serve. Even though its nature is different from the real-world railway infrastructure network, studying its robustness can be proven useful nonetheless, as it could allow companies to identify possible areas of improvement in their train routes.

Then, by using this model, I will show how the network handles in case of failure of an arbitray number of its nodes. I will present both theoretical results and numerical simulations, and I will compare them with previous works on a similar network. In order to best introduce the theoretical results, I will 
explain some concepts regarding network degree distribution, small world and scale free networks, node degree correlation and community subdivision.

Finally, I will present a summary of the result obtained and give ideas for further research.
